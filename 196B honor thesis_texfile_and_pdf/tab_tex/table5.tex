\begin{table}[H]

\caption{Characteristics of Candidates from Different Regions.}
\centering
\fontsize{8}{10}\selectfont
\resizebox{\columnwidth}{!}{
\begin{tabular}[t]{l c c c c c}

\hline\hline
\multicolumn{1}{l}{Region} &\multicolumn{1}{l}{Ph.D. Ranking}&\multicolumn{1}{l}{Undergraduate Ranking}&\multicolumn{1}{l}{Female} &\multicolumn{1}{l}{Master Degree} &\multicolumn{1}{l}{Observations}\\
\hline
East Asia & 21.3 & 141.6 & 0.36 & 0.65 & 141\\

Eastern Europe & 12.6 & 354.2 & 0.36 & 0.72 & 22\\

Latin America & 17.4 & 357.6 & 0.21 & 0.88 & 70\\

Middle East & 26 & 402 & 0.25 & 0.87 & 24\\

South Asia & 26.5 & 424.5 & 0.39 & 0.9 & 33\\

Western Europe & 13 & 166.9 & 0.29 & 0.89 & 68\\
\hline
\end{tabular}}
\vspace*{0.09cm}
\begin{minipage}{0.95\textwidth} 
{\footnotesize 
\textit{Note}: Table presents summary statistics of different characteristics by six regions. Ph.D. ranking ranges from 1 to 50, deterimined by U.S. news. Undergraduate ranking ranges from 1 to 700, surveyed by 2016 QS world ranking. Female is a dummy variable equal to 1 when the candidate is female. Master Degree is a dummy variable equal to 1 when the candidate has a graduate degree before Ph.D.}
\end{minipage}
\end{table}

